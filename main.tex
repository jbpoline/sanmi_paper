\documentclass{article}
\usepackage[utf8]{inputenc}
\usepackage{hyperref}
\usepackage{natbib}

\title{Brain Network Group Testing}
\author{--}
\date{June 2017}

\begin{document}

\maketitle
\section{Goals}
We assume that you are given resting state data from two groups of subjects -- a control group and a treatment group. We also assume that you have pre-computed networks from available resting state data.
Some references: \citep{varoquaux_brain_2010, belilovsky_learning_2016, belilovsky_testing_2016}

{\bf Main goal:}
\begin{itemize}
    \item Test if the {\em treatment} group data are well explained by the control group networks
    \item Estimate components for the treatment group
\end{itemize}

{\bf Preliminary goal:}
\begin{itemize}
    \item Test if the control group data are well explained by the pre-computed networks
    \item Estimate components for the control group such that they are similar to the pre-computed basis.
\end{itemize}
We plan to apply similar strategies for both problems.

\section{Model for control group}
Let $n$ denote the number of subjects, $s$ denote the number of brain regions, and $t$ denote the number of time points. Let $K_\alpha$ denote the number of known brain networks i.e. the pre-computed basis, and $K_\beta$ denote the number of un-known networks. Thus $K = K_\alpha + K_\beta$ is the total number of brain sub-networks.

The (known) brain sub-networks components are denoted by matrices $A_{s, k}$ shared across subjects, and the temporal weights are denoted by subject-specific parameters $B^n_{t, k}$. Similar;y, we denote unknown brain networks and weights by $C_{s, k}$ and $D^n_{s, k}$ respectively. We assume the within-group noise is IID Gaussian $E^{n}_{s,t} \sim N(0, 1)$. The resulting model for each sample is:
\begin{equation}
X^{n}_{s,t} = \sum_{k \in [K_\alpha]} A_{s, k} B^n_{t, k} + \sum_{k \in [K_\beta]} C_{s, k} D^n_{t, k} + E^{n}_{s,t}
\end{equation}

\subsection{Inference}
We seek to test if $K_\beta >0$ i.e. if the data are well explained by the known networks of not. {\bf Proposal:} Apply a likelihood ration test to compare $P(\mathbf{X}|K_\beta=0)$ vs $P(\mathbf{X}|K_\beta=K_0)$. More details TBD.

\subsection{Estimation}
The first step is to estimate the factors associated with known components $B^n_{t, k}, {k \in [K_\alpha]}$, assuming $A_{s, k}$ are known and fixed based on pre-extisting data. This can be solved using linear regression (possibly sparse linear regression.

The second steo us to estimagte the new components (if any) i.e. $C_{s, k}, D^n_{t, k}, {k \in [K_\beta]}$. We can solve this by computing the residual:
$$R^{n}_{s,t} =  X^{n}_{s,t} - \sum_{k \in [K_\alpha]} A_{s, k} B^n_{t, k} $$
Following this, we can apply the group dictionary level model 
%\footnote{ \hyperlink{http://nilearn.github.io/auto_examples/03_connectivity/plot_compare_resting_state_decomposition.html}{http://nilearn.github.io/auto_examples/03_connectivity/plot_compare_resting_state_decomposition.html}} to the residual.

Alternatively, we may be able to compute better estimates using a joint approach instead of separate estimation.
\newpage
\subsection*{References}
\bibliographystyle{plain}
\bibliography{references}

\end{document}

